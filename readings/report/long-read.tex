% use the answers clause to get answers to print; otherwise leave it out.
\documentclass{article}
%\documentclass[12pts, answers]{exam}
%\documentclass[12pts]{exam}
\RequirePackage{amssymb, amsfonts, amsmath, latexsym, verbatim, xspace, setspace}
\usepackage{graphicx}

\usepackage{minted}
 

% By default LaTeX uses large margins.  This doesn't work well on exams; problems
% end up in the "middle" of the page, reducing the amount of space for students
% to work on them.
\usepackage[top=1in,bottom=1in,right=1in,left=1in]{geometry}
\usepackage{enumerate}
\usepackage{color}
\usepackage{hyperref}
\usepackage{placeins}

% Here's where you edit the Class, Exam, Date, etc.
\newcommand{\class}{NPRE 412}
\newcommand{\term}{Spring 2025}
\newcommand{\assignment}{Book Report}
\newcommand{\duedate}{2025.05.06}
%\newcommand{\timelimit}{50 Minutes}

\newcommand{\nth}{n\ensuremath{^{\text{th}}} }
\newcommand{\ve}[1]{\ensuremath{\mathbf{#1}}}
\newcommand{\Macro}{\ensuremath{\Sigma}}
\newcommand{\vOmega}{\ensuremath{\hat{\Omega}}}

% For an exam, single spacing is most appropriate
\singlespacing
% \onehalfspacing
% \doublespacing

% For an exam, we generally want to turn off paragraph indentation
\parindent 0ex

%\unframedsolutions
\usepackage{bibentry}
\begin{document} 

% These commands set up the running header on the top of the exam pages
%\pagestyle{head}
%\firstpageheader{}{}{}
%\runningheader{\class}{\assignment\ - Page \thepage\ of \numpages}{Due \duedate}
%\runningheadrule

\class \hfill \term \\
\assignment \hfill Due \duedate (report) \& 2021.05.13 (presentation)\\
%\begin{flushright}
%\begin{tabular}{p{5in} r l}
%\end{tabular}
%\end{flushright}
\rule[1ex]{\textwidth}{.1pt}

%%%%%%%%%%%%%%%%%%%%%%%%%%%%%%%%%%%%%%%%%%%%%%%%%%%%%%%%%%%%%%%%%%%%%%%%%%%%%%%%%%%%%
%%%%%%%%%%%%%%%%%%%%%%%%%%%%%%%%%%%%%%%%%%%%%%%%%%%%%%%%%%%%%%%%%%%%%%%%%%%%%%%%%%%%%

In NPRE 412, a large part of your grade is earned with reading a compelling 
book about nuclear power, the nuclear fuel cycle, energy economics, or energy 
ethics. Reading, considering, and reporting on this reading is intended to 
build your technical writing skills, tie together the lessons of the course, 
and hone your existing critical thinking skills.

\section{Report}
In a signficant, sophisticated document, describe the content of the book and 
critically evaluate it. Consider the framing of topics, the approach to facts, 
the approach to emotions, and the purpose and audience of the book. Identify 
key lessons you learned from reading it.

\subsection{Abstract}
In a paragraph, summarize your report. This is best written last, as it covers the following sections in brief.

\subsection{Summary}
The report should begin with a summary of the book. Describe the narrative arc 
of the book, describe the main ideas and topics discussed in the book, and note 
the structure of the book. Provide an idea of the full scope of the content in 
the book. Your book report should have at least the following sections.

\subsection{Author and Audience}
Comment on the author's background and viewpoint on the topic as well as the 
target audience. 
Consider questions including, but not limited to:
\begin{itemize}
        \item Is the author qualified to write the book?
        \item Why did the author choose to write about this topic? 
        \item What was the main point (or points) that the author was trying to get across? 
        \item How can you tell what audience the author is writing for? 
        \item What did book reviewers think of this book online (e.g. on 
                Amazon, Goodreads)
        \item What did the media think of this book (e.g. Does it appear 
                in the New York Times Book Review)?
\end{itemize}


\subsection{Critical Technical Evaluation}
Comment on the technical content of the book and the book's approach to communicating it. 
Consider questions including, but not limited to:

\begin{itemize}
        \item Did the author present accurate information? 
	\item Can the information in the book be verified?
	\item Was the technical information presented thoroughly, or were important details left out? 
	\item Was the techical information easy to understand? 
	\item Did the author convince you of their views, or were there things you remain skeptical or critical of?
\end{itemize}

\subsection{Critical Non-technical Evaluation}
Comment on the non-technical content of the book and the book's approach to communicating it. 
This may include historic facts of a non-technical nature, personal naratives, and appeals to emotion.
Consider questions including, but not limited to:

\begin{itemize}
        \item Did the author present accurate information? 
	\item Can the information in the book be verified?
	\item Was the non-technical information presented thoroughly, or were important details left out? 
	\item Was the non-techical information easy to understand? 
	\item What emotions did you feel while reading the book?
	\item What emotions did the author likely intend to stir in the reader?
	\item Did the author convince you of their views, or were there things you remain skeptical or critical of?
\end{itemize}

\subsection{Conclusion}
Books can be long, dense, short, light, repetitive, frustrating, or illuminating. Give your general thoughts on the book.
\begin{itemize}
	\item How did your knowledge and opinions frame your reaction to the book? 
	\item Were any of your assumptions or deeply held opinions challenged during the reading of the book? 
	\item What was your overall opinion of the book?
	\item Would you recommend it to others? 
\end{itemize}

\subsection{References}
Include a bibliography. Your report should cite, at the very least, the book itself. Ideally, your previous sections might also reference related literature, primary sources, media, and other content that should appear in the bibliography, here.



\section{Presentation}
On the final exam day of class, please be prepared to present the content of 
this report in a very short presentation (5-10 minutes, depending on 
scheduling). You will be judged on your presentation style as well as ability 
to communicate the content of this report quickly.

\section{Report Formatting}
\subsection{General}
Please write a comprehensive, self-contained report.
                \begin{itemize}
                        \item It must be computer generated, not hand written 
                                {\color{red}\textbf{[-5\%]}}.
                        \item PDF must be submitted via classroom.github.com by 
                                the due date  {\color{red}\textbf{[-5\%]}}.
                \end{itemize}

\subsection{Content Formatting}
        \begin{itemize}
	\item Write this report like a narrative, with sections.
        \item    Report should be self-contained, do not repeat assignment text, do not 
                copy/paste the assignment itself  {\color{red}\textbf{[-5\%]}}.
        \item    Do not submit results as raw ``column of numbers'' data  {\color{red}\textbf{[-5\%]}}.
        \item    Do not include your source code in the report  {\color{red}\textbf{[-5\%]}}.
        \item    Snippets (small parts) of the source code are OK, if relevant. 
                Consider using the \LaTeX \texttt{minted} package for syntax highlighting, if 
                you're using \LaTeX.
        \item    Do not include commands typed in the prompt (Matlab, shell, compiler, etc) 
                 {\color{red}\textbf{[-5\%]}}.
        \item    Do not include extra plots/figures  {\color{red}\textbf{[-5\%]}}.
        \item    Additional figures that support the requested results are OK.
        \item   Report length should be less than 10 pages, if you exceed 10 pages, 
                you are probably doing something wrong, for example:
                \begin{itemize}
                        \item  1-3 pages for part 1 problem description, equations, derivation of solutions
                        \item  1-3 pages for part 2 problem description, results, discussion
                        \item  1-3 pages for part 3 problem description, results, discussion
                \end{itemize}
        \item    Obviously wrong solution  {\color{red}\textbf{[-5\%]}}.
        \item    Include well-formatted references  {\color{red}\textbf{[-5\%]}} 
\end{itemize}

\subsection{Formatting}
\begin{itemize}
        \item Cover page with your name, assignment title/number, course number, date {\color{red}\textbf{[-5\%]}}.
        \item     Include page numbers, except on the title/cover page  {\color{red}\textbf{[-5\%]}}.
        \item Report body has to start on page 1  {\color{red}\textbf{[-5\%]}}.
        \item    Use portrait orientation  {\color{red}\textbf{[-5\%]}}.
        \item    Landscape for a single page with large table/figure is OK.
        \item Plots, figures, and their labels must be formatted to be visible, 
                readable and differentiable on the printout  {\color{red}\textbf{[-5\%]}}.
        \item    Use only one font type and size for the main body of the report  {\color{red}\textbf{[-5\%]}}.
        \item    Do not use monospaced font for the report body  {\color{red}\textbf{[-5\%]}}.
\end{itemize}

\subsection{Equations}
Your book report might not include equations. But, perhaps it will.
\begin{itemize}
        \item Number each equation in a consistent way  {\color{red}\textbf{[-5\%]}}.
        \item    Equations should be numbered to the right of the equation  {\color{red}\textbf{[-5\%]}}.
        \item    Use notation consistent with the class lectures or textbook  {\color{red}\textbf{[-5\%]}}.
        \item    Typeset equations properly (e.g. Equation Editor, LaTeX, MathType, etc.), 
do not type them as unformatted text or inject them as grainy images.  {\color{red}\textbf{[-5\%]}}.
\end{itemize}

\section{Tables and Figures}
\begin{itemize}
        \item Number and label each table and figure in a consistent way  {\color{red}\textbf{[-5\%]}}.
        \item All figures should be captioned and should be referenced in the 
                text. 
        \item    Use proper labels for plots, figures, tables – title, axis, legend, units, 
etc.  {\color{red}\textbf{[-5\%]}}.
\item   Table title should be above the table, figure title should be below the 
figure  {\color{red}\textbf{[-5\%]}}.
\item   Titles, legends, labels must be of sufficient size and quality to be 
easily readable  {\color{red}\textbf{[-5\%]}}.
\item    Make units (e.g. time) on plots/figures understandable to humans  {\color{red}\textbf{[-5\%]}}, 
for example:
\begin{itemize}
        \item   if scale exceeds 100s of sec, change to min
        \item   if scale exceeds 100s of min, change to hours
        \item   if scale exceeds 100s of hours, change to days, etc…
        \item   If solution behavior is not visible on the plot because of the 
                scale, make another plot with a different scale (or log scale) 
                that clearly shows the solution behavior  {\color{red}\textbf{[-5\%]}}.
\end{itemize}
\item    Use sufficiently high quality figures such that they look smooth and sharp 
 {\color{red}\textbf{[-5\%]}}.
\begin{itemize}
\item    Screen shots are probably too low quality.
\item   \texttt{jpeg} and other lossy compression types are probably too low quality.
\item   High resolution and lossless vectorized image types are recommended.
        \end{itemize}
        \end{itemize}

        \section{Other}
  The purpose of the assignment is a comprehensive, self-contained,
consistently formatted report and a demonstration that you read and critically
contemplated the book assigned to you.  If you are not sure about what and how
much to include in the report, imagine that you have to grade it - make it
concise and easy to follow. I'm being picky because I want you to write good
reports.  The content and formatting rules are \emph{almost} universal.  

\end{document}
