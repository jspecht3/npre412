%        File: wk10.tex
%     Created: Sat Aug 20 01:00 PM 2016 C
% Last Change: Sat Aug 20 01:00 PM 2016 C
%

% use the answers clause to get answers to print; otherwise leave it out.
\documentclass[12pts]{exam}
%\documentclass[12pts]{exam}
\RequirePackage{amssymb, amsfonts, amsmath, latexsym, verbatim, xspace, setspace}

% By default LaTeX uses large margins.  This doesn't work well on exams; problems
% end up in the "middle" of the page, reducing the amount of space for students
% to work on them.
\usepackage[margin=1in]{geometry}
\usepackage{enumerate}

% Here's where you edit the Class, Exam, Date, etc.
\newcommand{\class}{NPRE 412}
\newcommand{\term}{Spring 2025}
\newcommand{\assignment}{Reading List Week 10}
\newcommand{\duedate}{2025.04.10}
%\newcommand{\timelimit}{50 Minutes}

\newcommand{\nth}{n\ensuremath{^{\text{th}}} }
\newcommand{\ve}[1]{\ensuremath{\mathbf{#1}}}
\newcommand{\Macro}{\ensuremath{\Sigma}}
\newcommand{\vOmega}{\ensuremath{\hat{\Omega}}}

% For an exam, single spacing is most appropriate
\singlespacing
% \onehalfspacing
% \doublespacing

% For an exam, we generally want to turn off paragraph indentation
\parindent 0ex

%\unframedsolutions
%\usepackage{bibentry}
\usepackage[hyphens]{url}
\usepackage[hidelinks]{hyperref}
\hypersetup{breaklinks=true}
\usepackage{splitbib}



\begin{category}{Required}
        \SBentries{tsoulfanidis_nuclear_2013}
        \SBentries{ayers_report_2012}
\end{category}
\begin{category}{Recommended}
\end{category}


\begin{document} 

% These commands set up the running header on the top of the exam pages
\pagestyle{head}
\firstpageheader{}{}{}
\runningheader{\class}{\assignment\ - Page \thepage\ of \numpages}{Due \duedate}
\runningheadrule

\begin{flushright}
                \class \hfill \term \\
                \assignment \hfill Due \duedate
\end{flushright}
\rule[1ex]{\textwidth}{.1pt}
\qformat{}

%%%%%%%%%%%%%%%%%%%%%%%%%%%%%%%%%%%%%%%%%%%%%%%%%%%%%%%%%%%%%%%%%%%%%%%%%%%%%%%%%%%%%
%%%%%%%%%%%%%%%%%%%%%%%%%%%%%%%%%%%%%%%%%%%%%%%%%%%%%%%%%%%%%%%%%%%%%%%%%%%%%%%%%%%%%

% ---------------------------------------------
\Urlmuskip=0mu plus 1mu\relax

\nocite{*}
\renewcommand\refname{}
\bibliographystyle{plainurl}
\bibliography{wk10}
\begin{questions}
        \question
\end{questions}


\end{document}

