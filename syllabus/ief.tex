%        File: ief.tex
%     Created: Wed Oct 05 08:00 AM 2016 C
% Last Change: Wed Oct 05 08:00 AM 2016 C
%

% use the answers clause to get answers to print; otherwise leave it out.
\documentclass[11pt,answers,addpoints]{exam}
%\documentclass[12pts]{exam}
\RequirePackage{amssymb, amsfonts, amsmath, latexsym, verbatim, xspace, setspace}

% By default LaTeX uses large margins.  This doesn't work well on exams; problems
% end up in the "middle" of the page, reducing the amount of space for students
% to work on them.
\usepackage[margin=1in]{geometry}
\usepackage{enumerate}
\usepackage{hyperref}

% Here's where you edit the Class, Exam, Date, etc.
\newcommand{\class}{NPRE412}
\newcommand{\term}{Spring 2025}
\newcommand{\assignment}{Informal Early Feedback}
\newcommand{\duedate}{2017.10.13}
%\newcommand{\timelimit}{50 Minutes}

\newcommand{\nth}{n\ensuremath{^{\text{th}}} }
\newcommand{\ve}[1]{\ensuremath{\mathbf{#1}}}
\newcommand{\Macro}{\ensuremath{\Sigma}}
\newcommand{\vOmega}{\ensuremath{\hat{\Omega}}}

% For an exam, single spacing is most appropriate
\singlespacing
% \onehalfspacing
% \doublespacing

% For an exam, we generally want to turn off paragraph indentation
\parindent 0ex

%\unframedsolutions
\usepackage{bibentry}
\begin{document} 


% These commands set up the running header on the top of the exam pages
\pagestyle{head}
\firstpageheader{}{}{}
\runningheader{\class}{\assignment\ - Page \thepage\ of \numpages}{Due \duedate}
\runningheadrule

\class \hfill \term \\
\assignment \hfill Due \duedate\\
\rule[1ex]{\textwidth}{.1pt}
%\hrulefill


%%%%%%%%%%%%%%%%%%%%%%%%%%%%%%%%%%%%%%%%%%%%%%%%%%%%%%%%%%%%%%%%%%%%%%%%%%%%%%%%%%%%%
%%%%%%%%%%%%%%%%%%%%%%%%%%%%%%%%%%%%%%%%%%%%%%%%%%%%%%%%%%%%%%%%%%%%%%%%%%%%%%%%%%%%%

\begin{itemize}
\item Thank you so much for your feeback. 
\item I will use this informal feedback to improve the course while you can still benefit. 
\item No names please.
\end{itemize}
\rule[1ex]{\textwidth}{.1pt}

% ---------------------------------------------
\begin{questions}
        % intro


        \question The level of the homework is :

\begin{checkboxes}
\choice too hard
\choice just right
\choice too easy
\end{checkboxes}



        \question The pace of the course is :
\begin{checkboxes}
\choice too fast
\choice just right
\choice too slow
\end{checkboxes}

        \question Have you been keeping up with the reading assignments? 

\begin{checkboxes}
\choice 100\%
\choice 90\%
\choice 80\%
\choice 70\%
\choice 60\%
\choice 50\%
\choice 25\%
\choice 0\%
\end{checkboxes}

	\question It's your turn to grade me. What grade would you give my teaching performance so far?

\begin{checkboxes}
\choice A
\choice B
\choice C
\choice D
\choice F
\end{checkboxes}

        \question How many hours do you spend on NPRE412 outside of class each week (homework/studying)?
\makeemptybox{10mm}
        \question What do I do currently that helps you learn? 
\makeemptybox{30mm}
        \question What can I do differently to help you learn better?
\makeemptybox{30mm}
        \question What is the biggest barrier to your learning in this class?
\makeemptybox{30mm}

\question You may use this space for any additional comments or feedback. Thank you. 
        \makeemptybox{80mm}

\end{questions}
\end{document}


